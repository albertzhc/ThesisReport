\chapter*{Abstract}\label{abstract}

Planetary gearboxes are widely used because of their large transmission ratio and high load resistance. Suddenly breakdown or failure could result in severe economic and safety losses. Thus an effective diagnosis method is of great importance. Despite the complexity of planetary gearboxes, extensive studies have been done on this area. But most of them are based on constant or narrow speed range applications. A wider speed range not only brings frequency change but also amplitude modulation caused by passage of resonance and torque change due to speed variation.

The experiment was carried out on UNSW planetary gearbox test rig. Three test sets are designed and one is performed on this stage. A promising diagnosis procedure is developed through literature review. It is firstly performing cepstrum modification to compensate resonance, then utilize order tracking to eliminate frequency modulation effects. At last TSA and Hilbert transform is performed to detect the fault frequency.

In test 1, the cepstrum exponential liftering method is proven effective to compensate resonance around 2 Hz of input shaft speed at 10 second. Since the vibration signal is dominated by the later high speed portion, signal from the last 10 seconds are processed and it effectively revealed the harmonics of planet gear fault frequency. As the impulse caused by gear fault is quite obvious in this test rig, Hilbert transform is more convenient and effective than TSA to detect the fault information. In the following stage, different fault types and larger speed fluctuation and wider speed range will be investigated. Then the effectiveness of this procedure will be validated.
