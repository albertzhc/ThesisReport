\chapter{Introduction}\label{ch:introduction}

The aim of this thesis report is to investigate a method to identify planetary gearbox fault by vibration analysis under varying speed condition. The planetary gearbox is widely used in the transmission system of vehicles, ships, helicopters and wind turbines because of its characters of large transmission ratio, high load resistance and ability to split input or output power. Besides these advantages, the complex structure of planetary gearbox brings difficulties to monitor its condition. Many methods have been developed in this area for planetary gearboxes running at constant speed. Such as time synchronous averaging, cepstrum analysis method and demodulation methods. But more planetary gearboxes are running in variable speed, for example, those in vehicles or wind turbines. The developed method could not be performed directly in this condition. A method suitable for variable speed planetary gearbox diagnosis will bring much wider application.

This article includes following parts:

\textbf{Chapter 1: Introduction} presents the aim of the project, brief background information and the structure of the report.

\textbf{Chapter 2: Literature Review} presents characteristics of planetary gearboxes and gear vibration signals. Possible vibration diagnosis techniques in constant and varying speed are investigated.

\textbf{Chapter 3: Methodology} presents the experiment equipment and prepared test sets. The data processing methods and diagnosis procedures are demonstrated.

\textbf{Chapter 4: Results and Discussion} shows the outcome of tests and discussed the advantages and short comes of the analysis methods.

\textbf{Chapter 5: Conclusion and Future Works} summarize the findings in this stage and discussed what to do in the following stage.



