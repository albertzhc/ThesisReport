\chapter{Literature Review}\label{ch:literature}

In this chapter, some basic concepts and important knowledge are provided. These concepts are 

\section{Planetary gearbox}

Planetary gearing or epicyclic gearing is a gear system typically consisting of four parts: sun gear, planet gear, ring gear and the planet carrier.

There are several ways of input-output method such as stationary ring gear, fixed carrier or no stationary part. The gear ratio of the planetary gearbox could be calculated as:

\begin{equation}
	N_s\omega_s + N_p\omega_p - (N_s + N_p)\omega_c = 0
\end{equation}
\begin{equation}
	N_r\omega_r - N_p\omega_p - (N_r - N_p)\omega_c = 0
\end{equation}

The following figure shows a typical planetary gearing system, which contains 3 planet gears.

\begin{figure}
	\centering
	\includegraphics{PGB}
	\caption{Planetary Gearbox Layout\cite{gearbox}}
	\label{simulationfigure}
\end{figure}

The characters of planetary gearbox make it suitable for large transmission ratio, high load and split input or output circumstances. So it is widely used in wind turbines, lathes, automobiles and helicopters. The widely appliance and tough working environment of planetary gearbox require it to be highly dependable. Failures of planetary gearbox may lead to huge economic losses as well as safety incidents. But the compact and complex structure on the other hand make it difficult to monitor its condition. Especially when the load and operating speed are varying. This project focuses on the varying speed conditions.




\section{Planetary Gearbox fault diagnosis}

Vibration of gear is caused by the geometric deviation of gears and teeth deformation under load. These two effects introduce a 'meshing error' or 'transmission error' (TE). \cite{VBCM}The transmission error could be divided into three types: unloaded static TE, loaded static TE and dynamic TE. The unloaded static TE could be measured under a very light load, and it is realised to be caused by the geometric deviation. The load static TE is introduced by the tooth deflection under a constant load torque. Dynamic TE is caused by the fluctuation of torque and transmission speed.


\subsection{Vibration generated by gear}

Based on the understanding of transmission error, vibration generated by gears is classified as follows: \cite{VBCM}

1) Mean effects for all tooth pairs

The mean effects here are the same for all tooth pairs. Torque varies when each pair of teeth mesh and cause vibration. Therefore it is dominated by harmonics of tooth-mesh frequencies. It could be sub divided into tree parts: 

	- Tooth deflection due to mean torque.
		
	- The mean part of initial profile errors resulting from manufacturing.	
	
	- Uniform wear over all teeth.

Uniform wear of teeth could increase friction force, which would results in higher harmonics of the gearmesh frequency.

2) Variation from the mean.

Variation from the mean could give rise to side bands of harmonics and maybe caused by:
\begin{itemize}
\item Slow variations, such as distortion and runout.		
\item Local faults, such as tooth spalls and root cracks.		
\item Random errors.		
\item Systematic erros.
\end{itemize}
Sidebands around the harmonics of gearmesh frequencies contain the gear fault information. The spaces between sidebands and harmonics shows which gear has fault, while the form of sidebands identify the type of fault. For example, local faults may give rise to a flat sideband spectrum, while distributed fault may inspire higher level but narrowly grouped sidebands. 
Due to limitation of time and resource, the main fault investigated in this project is Local fault, including tooth spalls and root cracks.
Separation of spalls and cracks is another important topic due to the reason that cracks could cause a much more rapid failure. Endo\cite{Endo} developed a finite element analysis method to investigate their difference. It was found that cracks at tooth root give a two-stage deviation of transmission error due to the reason that at first stage, faulty tooth together with a healthy tooth share the load and at the second stage, faulty tooth stands the load alone. Spalls on the other hand inspire one deviation of TE when mating tooth pass the spall. 

Comparing to fixed-axis gearbox, in which each gear rotates around a fixed centre, planetary gearbox has planet gears which rotate around not only their own centres but also the centre of sun gear. The transmission structure of planetary gearbox bring unique behaviours. \cite{review}
\begin{enumerate}
\item The planet gears are meshing simultaneously with sun gear and the ring gear. Part of the vibrations exited by different component and their different meshing phases could be neutralized or cancelled by each other.

\item The multiple vibration transmission paths are time-varing and load effected in planetary gearbox.It could attenuate the vibration signal of defective part and weaken the fault characteristics.

\item Differ from fixed-axis gear box, side bands apear in spectrum for both healthy and faulty planetary gearboxes and asymmetric about the tooth-mesh frequency. It may caused by multiple planet gears meshing with different phases.

\item Vibrations of low speed faulty components are easily masked and difficult to discover.
\end{enumerate}


\subsection{Diagnosis techniques}

As discussed in the former section, when gears are operating in good condition, the vibration signal tend to be stationary, containing gearmesh frequency and shafts rotating frequencies. When fault happens, the amplitude or frequency components change according to the fault types.
Plenty of diagnosis techniques are developed to seperate the faulty information from the original signal. \cite{practical}

\subsubsection{Statistic Indicators}

Time domain statistical indicators are carried out directly from the vibration signal. Some of them are scaled including peak value, peak to peak value, mean value, root mean squared value and variance.
Among which RMS (root mean squared) value, as its name suggests, is the root of the mean of the squared signal values. It represents the overall vibration level. It is calculated as:
\begin{equation}
	RMS = \sqrt{\frac{1}{N}\sum_{i=1}^n x_{i}^2}
\end{equation}

Variance indicates the power of the vibration, and its formula is:

\begin{equation}
	\sigma^2 = \frac{1}{N}\sum_{i=1}^n (x_{i} - \overline{x})^2
\end{equation}

There are also some useful unscaled indicators, including kurtosis, crest factor and pulse factor. Kurtosis is the fourth moment normalised by the square of the mean square of the vibration signal waveform and represents the amplitude of impulse energy. \cite{trending} 
\begin{equation}
	K = \frac{\frac{1}{N}\sum_{i=1}^n (x_{i} - \overline{x})^4}{RMS^4}
\end{equation}

Crest factor shows impulse energy in another form:
\begin{equation}
	C = \frac{x_{peak}}{RMS}
\end{equation}

Scaled indicators not only depend on the condition of the machine but also its running speed and load. Unscaled indicators have the benefit that they are independent of the running status.

\subsubsection{Time Synchronous Averaging}






\section{Variable Speed}















