\chapter{Literature Review}\label{ch:literature}

In this chapter, some basic concepts and important knowledge are provided. These concepts are 

\section{Planetary gearbox}

Planetary gearing or epicyclic gearing is a gear system typically consisting of four parts: sun gear, planet gear, ring gear and the planet carrier.

There are several ways of input-output method such as stationary ring gear, fixed carrier or no stationary part. The gear ratio of the of the planetary gearbox could be calculated as:

\begin{equation}
	N_s\omega_s + N_r\omega_r - (N_s + N_r)\omega_c
\end{equation}

The following figure shows a typical planetary gearing system, which contains 3 planet gears.


\begin{figure}
	\centering
	\includegraphics{PGB}
	\caption{Planetary Gearbox Layout\cite{gearbox}}
	\label{simulationfigure}
\end{figure}

The characters of planetary gearbox make it suitable for large transmission ratio, high load and split input or output circumstances. So it is widely used in wind turbines, lathes, automobiles and helicopters. The widely appliance and high load performed on planetary gearbox on the other hand require it highly dependable. But the compact and complax structure make it difficult to monitor its condition. Especaially when the load and operating speed are varying.











\section{Planetary Gearbox fault diagnosis}



\subsection{Vibration generated by gear}

1)	Normal operation

2)	Fault, spalls and cracks

\subsection{Diagnosis techniques}

1)	Standard Indicators

2)	wavelet

3)	Cepstrum

4)	TSA

5)	Order tracking

6)	Machine learning method


\section{Variable Speed}