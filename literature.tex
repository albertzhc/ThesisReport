\chapter{Literature Review}\label{ch:literature}

In this chapter, some basic concepts and important knowledge are provided. These concepts are 

\section{Planetary Gearbox}

Planetary gearing or epicyclic gearing is a gear system typically consisting of four parts: sun gear, planet gear, ring gear and the planet carrier.

There are several ways of input-output method such as stationary ring gear, fixed carrier or no stationary part. The gear ratio of the planetary gearbox could be calculated as:

\begin{equation}
	N_s\omega_s + N_p\omega_p - (N_s + N_p)\omega_c = 0
\end{equation}
\begin{equation}
	N_r\omega_r - N_p\omega_p - (N_r - N_p)\omega_c = 0
\end{equation}

The following figure shows a typical planetary gearing system, which contains 3 planet gears.

\begin{figure}
	\centering
	\includegraphics{PGB}
	\caption{Planetary gearbox layout \cite{gearbox}}
	\label{simulationfigure}
\end{figure}

The characters of planetary gearbox make it suitable for large transmission ratio, high load and split input or output circumstances. So it is widely used in wind turbines, lathes, automobiles and helicopters. The widely appliance and tough working environment of planetary gearbox require it to be highly dependable. Failures of planetary gearbox may lead to huge economic losses as well as safety incidents. But the compact and complex structure on the other hand make it difficult to monitor its condition. Especially when the load and operating speed are varying. This project focuses on the varying speed conditions.


\section{Planetary Gearbox Fault Diagnosis}

Vibration of gear is caused by the geometric deviation of gears and teeth deformation under load. These two effects introduce a 'meshing error' or 'transmission error' (TE) \cite{vbcm}. The transmission error could be divided into three types: unloaded static TE, loaded static TE and dynamic TE. The unloaded static TE could be measured under a very light load, and it is realised to be caused by the geometric deviation. The load static TE is introduced by the tooth deflection under a constant load torque. Dynamic TE is caused by the fluctuation of torque and transmission speed.


\subsection{Vibration generated by gear}

Based on the understanding of transmission error, vibration generated by gears is classified as follows \cite{vbcm}:
\begin{enumerate}
\item Mean effects for all tooth pairs

The mean effects here are the same for all tooth pairs. Torque varies when each pair of teeth mesh and cause vibration. Therefore it is dominated by harmonics of tooth-mesh frequencies. It could be sub divided into tree parts: 
\begin{itemize}
\item Tooth deflection due to mean torque.		
\item The mean part of initial profile errors resulting from manufacturing.		
\item Uniform wear over all teeth.
\end{itemize}
Uniform wear of teeth could increase friction force, which would results in higher harmonics of the gearmesh frequency.

\item Variation from the mean.

Variation from the mean could give rise to side bands of harmonics and maybe caused by:
\begin{itemize}
\item Slow variations, such as distortion and runout.		
\item Local faults, such as tooth spalls and root cracks.		
\item Random errors.		
\item Systematic erros.
\end{itemize}
Sidebands around the harmonics of gearmesh frequencies contain the gear fault information. The spaces between sidebands and harmonics shows which gear has fault, while the form of sidebands identify the type of fault. For example, local faults may give rise to a flat sideband spectrum, while distributed fault may inspire higher level but narrowly grouped sidebands. 
Due to limitation of time and resource, the main fault investigated in this project is Local fault, including tooth spalls and root cracks.
\end{enumerate}

Separation of spalls and cracks is another important topic due to the reason that cracks could cause a much more rapid failure. Endo \cite{Endo} developed a finite element analysis method to investigate their difference. It was found that cracks at tooth root give a two-stage deviation of transmission error due to the reason that at first stage, faulty tooth together with a healthy tooth share the load and at the second stage, faulty tooth stands the load alone. Spalls on the other hand inspire one deviation of TE when mating tooth pass the spall. 

\subsection{ Characteristic of planetary gearbox signals}

Comparing to fixed-axis gearbox, in which each gear rotates around a fixed centre, planetary gearbox has planet gears which rotate around not only their own centres but also the centre of sun gear. The transmission structure of planetary gearbox bring unique behaviours \cite{review}.
\begin{enumerate}
\item The planet gears are meshing simultaneously with sun gear and the ring gear. Part of the vibrations exited by different component and their different meshing phases could be neutralized or cancelled by each other.

\item The multiple vibration transmission paths are time-varing and load effected in planetary gearbox.It could attenuate the vibration signal of defective part and weaken the fault characteristics.

\item Differ from fixed-axis gear box, side bands apear in spectrum for both healthy and faulty planetary gearboxes and asymmetric about the tooth-mesh frequency. It may caused by multiple planet gears meshing with different phases.

\item Vibrations of low speed faulty components are easily masked and difficult to discover.
\end{enumerate}


\subsection{Diagnosis techniques}

As discussed in the former section, when gears are operating in good condition, the vibration signal tend to be stationary, containing gearmesh frequency and shafts rotating frequencies. When fault happens, the amplitude or frequency components change according to the fault types.
Plenty of diagnosis techniques are developed to seperate the faulty information from the original signal \cite{practical} .

\subsubsection{Statistic indicators}

Time domain statistical indicators are carried out directly from the vibration signal. Some of them are scaled including peak value, peak to peak value, mean value, root mean squared value and variance.
Among which RMS (root mean squared) value, as its name suggests, is the root of the mean of the squared signal values. It represents the overall vibration level. It is calculated as:
\begin{equation}
	RMS = \sqrt{\frac{1}{N}\sum_{i=1}^n x_{i}^2}
\end{equation}

Variance indicates the power of the vibration, and its formula is:

\begin{equation}
	\sigma^2 = \frac{1}{N}\sum_{i=1}^n (x_{i} - \overline{x})^2
\end{equation}

There are also some useful unscaled indicators, including kurtosis, crest factor and pulse factor. Kurtosis is the fourth moment normalised by the square of the mean square of the vibration signal waveform and represents the amplitude of impulse energy \cite{trending}.
\begin{equation}
	K = \frac{\frac{1}{N}\sum_{i=1}^n (x_{i} - \overline{x})^4}{RMS^4}
\end{equation}

Crest factor shows impulse energy in another form:
\begin{equation}
	C = \frac{x_{peak}}{RMS}
\end{equation}

Scaled indicators not only depend on the condition of the machine but also its running speed and load. Unscaled indicators have the benefit that they are independent of the running status.

\subsubsection{Time synchronous averaging}

TSA (Time synchronous averaging) is also a time domain signal processing method other than statistical indicators. It is implemented by averaging several segments of synchronousing signals together to extract the periodic signal from the backgound noises. It is calculated by:
\begin{equation}
	y_{a}(t) = \frac{1}{N}\sum_{n=0}^{N-1} y(t+nT)
\end{equation}

The implementation of TSA depend on the corresponding of sample signals. Which means to be averaged numbers at each point should have the same rotating phase. This is guaranteed by 'Order traking' method. When analysing vibration signals of the rotating machines, a slight fluctuation of the rotating speed could cause smearing of the frequecy components. Order tracking is taking use of a shaft encoder signal or a tacho signal to resample the original signals with constant time interval into constant phase intervaled signal. This method make sure that the samples have same number and starting phase during each revolution. It is recomment to always perform order tracking before implementing TSA even at a constant speed.

The TSA method is firstly introduced into gear fault diagnosis by Stewart \cite{stewart}, whose propose to abtain the residual signal by removing the periodic gearmeshing pattern, and use the kurtosis of the residual signal as the fault indicator. Peter McFadden \cite{mc2} enhanced this method by improvements of the TSA as well as order tracking operations. He proposed to select short section of signals which correspond to the vibration of one tooth and assembly together to the whole gear. Taking twice length of toothmesh period signal sections with Hanning window was recommended to improve the frequency spectra in his later papers \cite{mc3}. Time synchronous averaged signals for a fixed period, normally one or two rotation of the shaft, would illustrate the fault type of the gear. As shown in Figure 2.2, plot a) is a steady gearmesh vibration. In plot b), the signal is amplitude modulated, which maybe a consequence of misalignment. Plot c) is possibly because of significant wear of teeth surface, and plot d) may caused by cracked tooth.

\begin{figure}
	\centering
	\includegraphics{TSA}
	\caption{Time synchronous averaged signals \cite{chen}}
	\label{tsa}
\end{figure}


\subsubsection{Frequency-domain methods}

Time-domain methods such as statistical indicators is able to find whether the gear is in malfunction, but not sufficient to the fault type.Frequency-domain methods are more capable of this requirement. They are developed based on Fourier transform. Under normal operating condition, the spectrum is domained by shaft rotating and gearmesh frequency. Sidebands around them and their harmonics would identify the type of fault in some content as discussed in chapter 2.2.1.
Further than Fourier transform or its implementation for digital signal, DFT and FFT, demodulation techniques and cepstrum analysis are utilized to run the diagnosis.

As already known that the faulty information of gears ususally hide in the sidebands around gearmesh frequency and its harmonics, and these sidebands are mostly caused by amplitude and frequency modulation. Demodulation techniques such as Hilbert transform are effective to carry out the modulation signal. Thus it is able to investigate their faulty type and damage extent.
One example is introduced in \cite{mc4}. It use Hilbert transform to demodulate the second harmonic of gearmsh frequency of the TSA signal. As seen in Figure \ref{demodulation}, the phase modulation signal of the TSA signal clearly shows the cracked tooth. It is even able to be known which tooth is in fault with certain phase information.

\begin{figure}
	\centering
	\includegraphics{demodulation}
	\caption{Demodulation of the second harmonic of the gearmesh frequency for a cracked gear \cite{mc4}}
	\label{demodulation}
\end{figure}

Cepstrum analysis is another important method to investigate the sidebands and transfer function. Cepstrum is the inverse Fourier transform of a logarithmic spectrum, which is Fourier transformed from a time signal. It is then:

\begin{equation}
\hat{x}(\tau) = IF[log(X(f))]
\end{equation}

Where $X(f)$ is the spectrum signal.

\subsubsection{Time-frequency-domain methods}

Time-frequency domain methods is a combination of time and frequecy domains in one plot. Methods such as Wigner-Ville distribution and wavelets analysis are studied recently as another choice in vibration diagnositics. Meltzer \textit{et al.} \cite{mel1} \cite{mel2} conducted this method on an automobile planetary gearbox.

\subsubsection{Other methods}

Some other effective methods exist in this area, for example modeling method and intelligent method.

...

\section{Fault Diagnosis under Variable Speed Condtions}

Most of the methods mentioned in the former part are developed for constant speed and load conditions. When there are minor speed fluctuations, performing of order tracking would eliminate this effect. But when it comes to wider speed range, as much as $\pm$ 30 percent, besides frequency modulation, amplitude modulation caused by passage of resonances, frequency response function and variation of torque comes into consideration. These effects could not be compensated by order tracking. Then the residual signal of TSA under these conditions would include much of the amplitude variation of the deterministic part as well as the random noise \cite{varyspeed}. One possible method is to find sections with relatively steady speed. Randall \cite{varyspeed} surgested another way to perform pre-processing of cepstrum modification to compensate the resonance part before order tracking and TSA.

\subsection{Cepstrum resonance compensation}

The cepstrum is obtained by inverse Fourier transform of the log spectrum. If the log spectrum includes phase part, it is complex cepstrum. If the log spectrum only has amplitude, then it is real or power cepstrum. The logarithmic conversion of the spectrum transform the multiplication relationship between the forcing function and transfer function into an additive one \cite{vbcm}. It is also more evident to see the families of harmonics and sidebands in the log spectrum. Due to these characteristics, cepstrum analysis is able to be applied for collecting uniformly spaced sidebands and separating forcing functions from transfer functions.

In the variable speed gearbox diagnosis application, cepstrum is utilized to perform resonance compensation. Randall indicated in \cite{resonance} that low-quefrency region of the cepstrum is contributed by both excitation and transfer function, while the high -quefrency region is domainated by forcing function. So it is able to use liftering in cepstrum to remove the forcing function part, or on the contrary remove the transfer function component. In \cite{varyspeed}, modal information is obtained by applying exponential window  to lifter the cepstrum. Removing the modal part in cepstrum and transfer back to spectrum and further to time domain, it should be able to seen a more uniform waveform. This means the amplitude variation due to passing resonances is compensated.

\subsection{Ordertracking and TSA}

Followed by the pre-processing of resonance compensation, 'order tracking' and TSA is now able to be performed. These two method have already been discussed in chapter 2.2.3. Cepstrum modification removed the amplitude modulation caused by resonance. The next step is to deal with the phase modulation caused by varying speed. 'Order tracking' is a phase demodulation technique based on a reference phase-locked signal. It resample the constant time interval signal to a constant phase increment signal. 'Order tracking' is used to avoid smearing of frequency component in its spectrem. Only after the application of order tracking, it is able to perform TSA.

As discussed in the former part, TSA is averaging together many signal segments corresponding to same period of synchronising signal. It is always performed for one or two rotation of the shaft. The averaged signal, as seen in Figures 2.2 and 2.3 would illustrate the fault in gears. A further investigation could be performed on the TSA or its residual signal. For example, the McFadden method /cite{McFadden} perform a more complicated TSA and check the Kurtosis of the residual signal. The Hilbert transform technique \cite{mc4} could be performed to examine the amplitude and phase modulation signal as shown in Figure 2.3.

\section{Summary}

By reviewing of the literatures above, it is aknowledged that vibration analysis of gearboxes has been widely researched. Planty of methods and techniques have been developed in this area. But most of these techniques are focused on fix axis gearboxes and constant speed conditions. Most of these methods fail to work for planetary gearboxes in varying speed conditions, which would be much more universal and widely performed. The complexity of planetary gearbox and amplitude modulation due to speed passing resonance bring a lot of challenge to this task. The suggested process is:

\begin{enumerate}
	\item Perform cepstrum modification for resonance compensation;
	
	\item Order tracking to compensate the speed variation;
	
	\item TSA to investigate the fault.
	
\end{enumerate}















